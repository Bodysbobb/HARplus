% Options for packages loaded elsewhere
\PassOptionsToPackage{unicode}{hyperref}
\PassOptionsToPackage{hyphens}{url}
%
\documentclass[
]{article}
\usepackage{amsmath,amssymb}
\usepackage{iftex}
\ifPDFTeX
  \usepackage[T1]{fontenc}
  \usepackage[utf8]{inputenc}
  \usepackage{textcomp} % provide euro and other symbols
\else % if luatex or xetex
  \usepackage{unicode-math} % this also loads fontspec
  \defaultfontfeatures{Scale=MatchLowercase}
  \defaultfontfeatures[\rmfamily]{Ligatures=TeX,Scale=1}
\fi
\usepackage{lmodern}
\ifPDFTeX\else
  % xetex/luatex font selection
\fi
% Use upquote if available, for straight quotes in verbatim environments
\IfFileExists{upquote.sty}{\usepackage{upquote}}{}
\IfFileExists{microtype.sty}{% use microtype if available
  \usepackage[]{microtype}
  \UseMicrotypeSet[protrusion]{basicmath} % disable protrusion for tt fonts
}{}
\makeatletter
\@ifundefined{KOMAClassName}{% if non-KOMA class
  \IfFileExists{parskip.sty}{%
    \usepackage{parskip}
  }{% else
    \setlength{\parindent}{0pt}
    \setlength{\parskip}{6pt plus 2pt minus 1pt}}
}{% if KOMA class
  \KOMAoptions{parskip=half}}
\makeatother
\usepackage{xcolor}
\usepackage[margin=1in]{geometry}
\usepackage{color}
\usepackage{fancyvrb}
\newcommand{\VerbBar}{|}
\newcommand{\VERB}{\Verb[commandchars=\\\{\}]}
\DefineVerbatimEnvironment{Highlighting}{Verbatim}{commandchars=\\\{\}}
% Add ',fontsize=\small' for more characters per line
\usepackage{framed}
\definecolor{shadecolor}{RGB}{248,248,248}
\newenvironment{Shaded}{\begin{snugshade}}{\end{snugshade}}
\newcommand{\AlertTok}[1]{\textcolor[rgb]{0.94,0.16,0.16}{#1}}
\newcommand{\AnnotationTok}[1]{\textcolor[rgb]{0.56,0.35,0.01}{\textbf{\textit{#1}}}}
\newcommand{\AttributeTok}[1]{\textcolor[rgb]{0.13,0.29,0.53}{#1}}
\newcommand{\BaseNTok}[1]{\textcolor[rgb]{0.00,0.00,0.81}{#1}}
\newcommand{\BuiltInTok}[1]{#1}
\newcommand{\CharTok}[1]{\textcolor[rgb]{0.31,0.60,0.02}{#1}}
\newcommand{\CommentTok}[1]{\textcolor[rgb]{0.56,0.35,0.01}{\textit{#1}}}
\newcommand{\CommentVarTok}[1]{\textcolor[rgb]{0.56,0.35,0.01}{\textbf{\textit{#1}}}}
\newcommand{\ConstantTok}[1]{\textcolor[rgb]{0.56,0.35,0.01}{#1}}
\newcommand{\ControlFlowTok}[1]{\textcolor[rgb]{0.13,0.29,0.53}{\textbf{#1}}}
\newcommand{\DataTypeTok}[1]{\textcolor[rgb]{0.13,0.29,0.53}{#1}}
\newcommand{\DecValTok}[1]{\textcolor[rgb]{0.00,0.00,0.81}{#1}}
\newcommand{\DocumentationTok}[1]{\textcolor[rgb]{0.56,0.35,0.01}{\textbf{\textit{#1}}}}
\newcommand{\ErrorTok}[1]{\textcolor[rgb]{0.64,0.00,0.00}{\textbf{#1}}}
\newcommand{\ExtensionTok}[1]{#1}
\newcommand{\FloatTok}[1]{\textcolor[rgb]{0.00,0.00,0.81}{#1}}
\newcommand{\FunctionTok}[1]{\textcolor[rgb]{0.13,0.29,0.53}{\textbf{#1}}}
\newcommand{\ImportTok}[1]{#1}
\newcommand{\InformationTok}[1]{\textcolor[rgb]{0.56,0.35,0.01}{\textbf{\textit{#1}}}}
\newcommand{\KeywordTok}[1]{\textcolor[rgb]{0.13,0.29,0.53}{\textbf{#1}}}
\newcommand{\NormalTok}[1]{#1}
\newcommand{\OperatorTok}[1]{\textcolor[rgb]{0.81,0.36,0.00}{\textbf{#1}}}
\newcommand{\OtherTok}[1]{\textcolor[rgb]{0.56,0.35,0.01}{#1}}
\newcommand{\PreprocessorTok}[1]{\textcolor[rgb]{0.56,0.35,0.01}{\textit{#1}}}
\newcommand{\RegionMarkerTok}[1]{#1}
\newcommand{\SpecialCharTok}[1]{\textcolor[rgb]{0.81,0.36,0.00}{\textbf{#1}}}
\newcommand{\SpecialStringTok}[1]{\textcolor[rgb]{0.31,0.60,0.02}{#1}}
\newcommand{\StringTok}[1]{\textcolor[rgb]{0.31,0.60,0.02}{#1}}
\newcommand{\VariableTok}[1]{\textcolor[rgb]{0.00,0.00,0.00}{#1}}
\newcommand{\VerbatimStringTok}[1]{\textcolor[rgb]{0.31,0.60,0.02}{#1}}
\newcommand{\WarningTok}[1]{\textcolor[rgb]{0.56,0.35,0.01}{\textbf{\textit{#1}}}}
\usepackage{graphicx}
\makeatletter
\def\maxwidth{\ifdim\Gin@nat@width>\linewidth\linewidth\else\Gin@nat@width\fi}
\def\maxheight{\ifdim\Gin@nat@height>\textheight\textheight\else\Gin@nat@height\fi}
\makeatother
% Scale images if necessary, so that they will not overflow the page
% margins by default, and it is still possible to overwrite the defaults
% using explicit options in \includegraphics[width, height, ...]{}
\setkeys{Gin}{width=\maxwidth,height=\maxheight,keepaspectratio}
% Set default figure placement to htbp
\makeatletter
\def\fps@figure{htbp}
\makeatother
\setlength{\emergencystretch}{3em} % prevent overfull lines
\providecommand{\tightlist}{%
  \setlength{\itemsep}{0pt}\setlength{\parskip}{0pt}}
\setcounter{secnumdepth}{-\maxdimen} % remove section numbering
\ifLuaTeX
  \usepackage{selnolig}  % disable illegal ligatures
\fi
\usepackage{bookmark}
\IfFileExists{xurl.sty}{\usepackage{xurl}}{} % add URL line breaks if available
\urlstyle{same}
\hypersetup{
  pdftitle={Enhanced Processing of GEMPACK .HAR and .SL4 Files},
  pdfauthor={Pattawee Puangchit},
  hidelinks,
  pdfcreator={LaTeX via pandoc}}

\title{Enhanced Processing of GEMPACK .HAR and .SL4 Files}
\author{Pattawee Puangchit}
\date{2025-02-18}

\begin{document}
\maketitle

{
\setcounter{tocdepth}{2}
\tableofcontents
}
\section{Overview of HARplus}\label{overview-of-harplus}

The \texttt{HARplus} package enhances GEMPACK users' experience by
streamlining \texttt{.har} and \texttt{.sl4} file processing. It
efficiently extracts, organizes, and manages dimension structures,
ensuring consistency, optimized memory usage, and simplified data
manipulation. The package enables fast data merging, aggregation, and
transformation while maintaining a consistent interface across GEMPACK
file types. Users can easily pivot data into familiar formats and export
results in various formats.

A key feature of \texttt{HARplus} is its \textbf{flexible subtotal level
handling}, allowing users to selectively retain \texttt{"TOTAL"} values,
decomposed components, or both. This ensures precise data extraction for
various economic modeling needs without unnecessary redundancy.

\section{Introduction}\label{introduction}

This vignette covers key functions of \texttt{HARplus}, highlighting its
ability to handle multiple inputs for efficient data processing.
Designed for economic modelers and GEMPACK practitioners, it simplifies
working with multiple datasets and enhances analytical workflows.

Sample data used in this vignette is obtained from the GTAPv7 model and
utilizes publicly available data from the GTAP 9 database. For more
details, refer to the
\href{https://www.gtap.agecon.purdue.edu/databases/archives.asp}{GTAP
Database Archive}.

\section{Acknowledgement}\label{acknowledgement}

The development of \texttt{HARplus} builds upon the foundational work of
the \texttt{HARr} package. I sincerely acknowledge and appreciate the
contributions of Maros Ivanic, whose \texttt{HARr} package served as the
baseline for this development, advancing efficient data handling for
GEMPACK users.

\section{Installation and Loading Example
Data}\label{installation-and-loading-example-data}

Before proceeding, ensure that \texttt{HARplus} is installed and loaded:

\begin{Shaded}
\begin{Highlighting}[]
\NormalTok{devtools}\SpecialCharTok{::}\FunctionTok{install}\NormalTok{(}\StringTok{"D:/GitHub/HARplus"}\NormalTok{)}
\CommentTok{\#\textgreater{} }
\CommentTok{\#\textgreater{} {-}{-} R CMD build {-}{-}{-}{-}{-}{-}{-}{-}{-}{-}{-}{-}{-}{-}{-}{-}{-}{-}{-}{-}{-}{-}{-}{-}{-}{-}{-}{-}{-}{-}{-}{-}{-}{-}{-}{-}{-}{-}{-}{-}{-}{-}{-}{-}{-}{-}{-}{-}{-}{-}{-}{-}{-}{-}{-}{-}{-}{-}{-}{-}{-}{-}{-}{-}{-}}
\CommentTok{\#\textgreater{}          checking for file \textquotesingle{}D:\textbackslash{}GitHub\textbackslash{}HARplus/DESCRIPTION\textquotesingle{} ...  v  checking for file \textquotesingle{}D:\textbackslash{}GitHub\textbackslash{}HARplus/DESCRIPTION\textquotesingle{}}
\CommentTok{\#\textgreater{}       {-}  preparing \textquotesingle{}HARplus\textquotesingle{}: (886ms)}
\CommentTok{\#\textgreater{}    checking DESCRIPTION meta{-}information ...  v  checking DESCRIPTION meta{-}information}
\CommentTok{\#\textgreater{}       {-}  excluding invalid files}
\CommentTok{\#\textgreater{}    Subdirectory \textquotesingle{}R\textquotesingle{} contains invalid file names:}
\CommentTok{\#\textgreater{}      \textquotesingle{}pivot\_output.xlsx\textquotesingle{}}
\CommentTok{\#\textgreater{}       {-}  checking for LF line{-}endings in source and make files and shell scripts}
\CommentTok{\#\textgreater{}   {-}  checking for empty or unneeded directories}
\CommentTok{\#\textgreater{}      Removed empty directory \textquotesingle{}HARplus/man\textquotesingle{}}
\CommentTok{\#\textgreater{}      NB: this package now depends on R (\textgreater{}=        NB: this package now depends on R (\textgreater{}= 3.5.0)}
\CommentTok{\#\textgreater{}        }\AlertTok{WARNING}\CommentTok{: Added dependency on R \textgreater{}= 3.5.0 because serialized objects in}
\CommentTok{\#\textgreater{}      serialize/load version 3 cannot be read in older versions of R.}
\CommentTok{\#\textgreater{}      File(s) containing such objects:}
\CommentTok{\#\textgreater{}        \textquotesingle{}HARplus/R/output\_directory/rds/sl4\_data1\_pca.rds\textquotesingle{}}
\CommentTok{\#\textgreater{}        \textquotesingle{}HARplus/R/output\_directory/rds/sl4\_data1\_qo.rds\textquotesingle{}}
\CommentTok{\#\textgreater{}        \textquotesingle{}HARplus/vignettes/output\_directory/rds/sl4\_data1\_pca.rds\textquotesingle{}}
\CommentTok{\#\textgreater{}        \textquotesingle{}HARplus/vignettes/output\_directory/rds/sl4\_data1\_qo.rds\textquotesingle{}}
\CommentTok{\#\textgreater{}   {-}  building \textquotesingle{}HARplus\_0.0.0.9000.tar.gz\textquotesingle{}}
\CommentTok{\#\textgreater{}      }
\CommentTok{\#\textgreater{} Running "D:/Program Files/R{-}4.4.2/bin/x64/Rcmd.exe" INSTALL \textbackslash{}}
\CommentTok{\#\textgreater{}   "C:\textbackslash{}temp\textbackslash{}RtmpcTLmFf/HARplus\_0.0.0.9000.tar.gz" {-}{-}install{-}tests }
\CommentTok{\#\textgreater{} * installing to library \textquotesingle{}C:/Users/b\_pat/AppData/Local/R/win{-}library/4.4\textquotesingle{}}
\CommentTok{\#\textgreater{} * installing *source* package \textquotesingle{}HARplus\textquotesingle{} ...}
\CommentTok{\#\textgreater{} ** using staged installation}
\CommentTok{\#\textgreater{} ** R}
\CommentTok{\#\textgreater{} ** inst}
\CommentTok{\#\textgreater{} ** byte{-}compile and prepare package for lazy loading}
\CommentTok{\#\textgreater{} ** help}
\CommentTok{\#\textgreater{} No man pages found in package  \textquotesingle{}HARplus\textquotesingle{} }
\CommentTok{\#\textgreater{} *** installing help indices}
\CommentTok{\#\textgreater{} ** building package indices}
\CommentTok{\#\textgreater{} ** installing vignettes}
\CommentTok{\#\textgreater{} ** testing if installed package can be loaded from temporary location}
\CommentTok{\#\textgreater{} ** testing if installed package can be loaded from final location}
\CommentTok{\#\textgreater{} ** testing if installed package keeps a record of temporary installation path}
\CommentTok{\#\textgreater{} * DONE (HARplus)}
\FunctionTok{library}\NormalTok{(HARplus)}
\end{Highlighting}
\end{Shaded}

\section{\texorpdfstring{Loading Data \texttt{.HAR} and
\texttt{.SL4}}{Loading Data .HAR and .SL4}}\label{loading-data-.har-and-.sl4}

First, load the example data using the following commands
\texttt{\textless{}load\_harx\textgreater{}} and
\texttt{\textless{}load\_sl4x\textgreater{}}:

\begin{Shaded}
\begin{Highlighting}[]
\CommentTok{\# Paths to the .har files}
\NormalTok{har\_path1 }\OtherTok{\textless{}{-}} \FunctionTok{system.file}\NormalTok{(}\StringTok{"extdata"}\NormalTok{, }\StringTok{"TAR10{-}WEL.har"}\NormalTok{, }\AttributeTok{package =} \StringTok{"HARplus"}\NormalTok{)}
\NormalTok{har\_path2 }\OtherTok{\textless{}{-}} \FunctionTok{system.file}\NormalTok{(}\StringTok{"extdata"}\NormalTok{, }\StringTok{"SUBT10{-}WEL.har"}\NormalTok{, }\AttributeTok{package =} \StringTok{"HARplus"}\NormalTok{)}

\CommentTok{\# Paths to the .sl4 files}
\NormalTok{sl4\_path1 }\OtherTok{\textless{}{-}} \FunctionTok{system.file}\NormalTok{(}\StringTok{"extdata"}\NormalTok{, }\StringTok{"TAR10.sl4"}\NormalTok{, }\AttributeTok{package =} \StringTok{"HARplus"}\NormalTok{)}
\NormalTok{sl4\_path2 }\OtherTok{\textless{}{-}} \FunctionTok{system.file}\NormalTok{(}\StringTok{"extdata"}\NormalTok{, }\StringTok{"SUBT10.sl4"}\NormalTok{, }\AttributeTok{package =} \StringTok{"HARplus"}\NormalTok{)}

\CommentTok{\# Load the .har files using load\_harx()}
\NormalTok{har\_data1 }\OtherTok{\textless{}{-}} \FunctionTok{load\_harx}\NormalTok{(har\_path1)}
\NormalTok{har\_data2 }\OtherTok{\textless{}{-}} \FunctionTok{load\_harx}\NormalTok{(har\_path2)}

\CommentTok{\# Load the .sl4 files using load\_sl4x()}
\NormalTok{sl4\_data1 }\OtherTok{\textless{}{-}} \FunctionTok{load\_sl4x}\NormalTok{(sl4\_path1)}
\NormalTok{sl4\_data2 }\OtherTok{\textless{}{-}} \FunctionTok{load\_sl4x}\NormalTok{(sl4\_path2)}
\end{Highlighting}
\end{Shaded}

\section{Extracting Data}\label{extracting-data}

This package allows two main ways to enhance the GEMPACK user experience
and improve user-friendliness when extracting data:\\
1. By selecting the variable/header name using
\texttt{\textless{}get\_data\_by\_var\textgreater{}}.\\
2. By using dimension patterns (e.g., \texttt{REG*COMM}) with
\texttt{\textless{}get\_data\_by\_dims\textgreater{}}.

Both functions have similar options:\\
- Extract a single variable, multiple variables, or all variables (NULL
or ``ALL'') from the datasets.\\
- Extract data from a single dataset or multiple datasets.\\
- Rename the experiment (i.e., dataset) name, which is displayed in the
\textbf{Experiment} column (this column is automatically added to handle
multiple dataset merges). By default, the name is the dataset name.\\
- Reporting data level (all, only total, only subtotals). - Rename
dimension names, which will be used as column names.\\
- Merge data across multiple datasets within the same patterns.\\
- For \texttt{\textless{}get\_data\_by\_var\textgreater{}}, variables
with the same name (e.g., \texttt{"qo"}) from different experiments
(\texttt{exp1} and \texttt{exp2}) will be merged into a single final
dataframe, with the input source identified in the \textbf{Experiment}
column.\\
- Similarly, for \texttt{\textless{}get\_data\_by\_dims\textgreater{}},
data with matching dimension patterns (e.g., \texttt{REG*COMM}) will be
merged.

Let's start with the simplest example as follows:

\begin{Shaded}
\begin{Highlighting}[]
\CommentTok{\# Extract data for a single variable}
\NormalTok{data\_qo }\OtherTok{\textless{}{-}} \FunctionTok{get\_data\_by\_var}\NormalTok{(}\StringTok{"qo"}\NormalTok{, sl4\_data1)}
\FunctionTok{print}\NormalTok{(}\FunctionTok{head}\NormalTok{(data\_qo[[}\StringTok{"sl4\_data1"}\NormalTok{]][[}\StringTok{"qo"}\NormalTok{]], }\DecValTok{4}\NormalTok{))}
\CommentTok{\#\textgreater{}          ACTS     REG Subtotal     Value Variable Dimension Experiment}
\CommentTok{\#\textgreater{} 1 GrainsCrops Oceania    TOTAL {-}5.283822       qo  ACTS*REG  sl4\_data1}
\CommentTok{\#\textgreater{} 2    MeatLstk Oceania    TOTAL {-}8.473134       qo  ACTS*REG  sl4\_data1}
\CommentTok{\#\textgreater{} 3  Extraction Oceania    TOTAL {-}1.617209       qo  ACTS*REG  sl4\_data1}
\CommentTok{\#\textgreater{} 4    ProcFood Oceania    TOTAL {-}3.570224       qo  ACTS*REG  sl4\_data1}

\CommentTok{\# Extract multiple variables from multiple datasets}
\NormalTok{data\_multiple }\OtherTok{\textless{}{-}} \FunctionTok{get\_data\_by\_var}\NormalTok{(}\FunctionTok{c}\NormalTok{(}\StringTok{"qo"}\NormalTok{, }\StringTok{"qgdp"}\NormalTok{), sl4\_data1, sl4\_data2)}
\FunctionTok{print}\NormalTok{(}\FunctionTok{head}\NormalTok{(data\_multiple[[}\StringTok{"sl4\_data1"}\NormalTok{]][[}\StringTok{"qo"}\NormalTok{]], }\DecValTok{4}\NormalTok{))}
\CommentTok{\#\textgreater{}          ACTS     REG Subtotal     Value Variable Dimension Experiment}
\CommentTok{\#\textgreater{} 1 GrainsCrops Oceania    TOTAL {-}5.283822       qo  ACTS*REG  sl4\_data1}
\CommentTok{\#\textgreater{} 2    MeatLstk Oceania    TOTAL {-}8.473134       qo  ACTS*REG  sl4\_data1}
\CommentTok{\#\textgreater{} 3  Extraction Oceania    TOTAL {-}1.617209       qo  ACTS*REG  sl4\_data1}
\CommentTok{\#\textgreater{} 4    ProcFood Oceania    TOTAL {-}3.570224       qo  ACTS*REG  sl4\_data1}
\FunctionTok{print}\NormalTok{(}\FunctionTok{head}\NormalTok{(data\_multiple[[}\StringTok{"sl4\_data2"}\NormalTok{]][[}\StringTok{"qo"}\NormalTok{]], }\DecValTok{4}\NormalTok{))}
\CommentTok{\#\textgreater{}          ACTS     REG Subtotal     Value Variable Dimension Experiment}
\CommentTok{\#\textgreater{} 1 GrainsCrops Oceania    TOTAL {-}5.283822       qo  ACTS*REG  sl4\_data2}
\CommentTok{\#\textgreater{} 2    MeatLstk Oceania    TOTAL {-}8.473134       qo  ACTS*REG  sl4\_data2}
\CommentTok{\#\textgreater{} 3  Extraction Oceania    TOTAL {-}1.617209       qo  ACTS*REG  sl4\_data2}
\CommentTok{\#\textgreater{} 4    ProcFood Oceania    TOTAL {-}3.570224       qo  ACTS*REG  sl4\_data2}

\CommentTok{\# Extract all variables separately from multiple datasets}
\NormalTok{data\_list }\OtherTok{\textless{}{-}} \FunctionTok{get\_data\_by\_var}\NormalTok{(}\ConstantTok{NULL}\NormalTok{, sl4\_data1, sl4\_data2)}
\FunctionTok{print}\NormalTok{(}\FunctionTok{names}\NormalTok{(data\_list))}
\CommentTok{\#\textgreater{} [1] "sl4\_data1" "sl4\_data2"}

\CommentTok{\# Extract all variables and merge the same variable from multiple datasets}
\NormalTok{data\_all }\OtherTok{\textless{}{-}} \FunctionTok{get\_data\_by\_var}\NormalTok{(}\ConstantTok{NULL}\NormalTok{, sl4\_data1, sl4\_data2, }\AttributeTok{merge\_data =} \ConstantTok{TRUE}\NormalTok{)}
\FunctionTok{print}\NormalTok{(}\FunctionTok{names}\NormalTok{(data\_all))}
\CommentTok{\#\textgreater{} [1] "merged"}

\CommentTok{\# Return all value levels}
\NormalTok{data\_all }\OtherTok{\textless{}{-}} \FunctionTok{get\_data\_by\_var}\NormalTok{(}\StringTok{"qo"}\NormalTok{, sl4\_data1, sl4\_data2, }\AttributeTok{subtotal\_level =} \ConstantTok{TRUE}\NormalTok{)}
\FunctionTok{print}\NormalTok{(}\FunctionTok{head}\NormalTok{(data\_all[[}\StringTok{"sl4\_data1"}\NormalTok{]][[}\StringTok{"qo"}\NormalTok{]], }\DecValTok{4}\NormalTok{))}
\CommentTok{\#\textgreater{}          ACTS     REG Subtotal     Value Variable Dimension Experiment}
\CommentTok{\#\textgreater{} 1 GrainsCrops Oceania    TOTAL {-}5.283822       qo  ACTS*REG  sl4\_data1}
\CommentTok{\#\textgreater{} 2    MeatLstk Oceania    TOTAL {-}8.473134       qo  ACTS*REG  sl4\_data1}
\CommentTok{\#\textgreater{} 3  Extraction Oceania    TOTAL {-}1.617209       qo  ACTS*REG  sl4\_data1}
\CommentTok{\#\textgreater{} 4    ProcFood Oceania    TOTAL {-}3.570224       qo  ACTS*REG  sl4\_data1}

\CommentTok{\# Return only TOTAL, drop subtotal}
\NormalTok{data\_total }\OtherTok{\textless{}{-}} \FunctionTok{get\_data\_by\_var}\NormalTok{(}\StringTok{"qo"}\NormalTok{, sl4\_data1, sl4\_data2, }\AttributeTok{subtotal\_level =} \ConstantTok{FALSE}\NormalTok{)}
\FunctionTok{print}\NormalTok{(}\FunctionTok{head}\NormalTok{(data\_total[[}\StringTok{"sl4\_data1"}\NormalTok{]][[}\StringTok{"qo"}\NormalTok{]], }\DecValTok{4}\NormalTok{))}
\CommentTok{\#\textgreater{}          ACTS     REG Subtotal     Value Variable Dimension Experiment}
\CommentTok{\#\textgreater{} 1 GrainsCrops Oceania    TOTAL {-}5.283822       qo  ACTS*REG  sl4\_data1}
\CommentTok{\#\textgreater{} 2    MeatLstk Oceania    TOTAL {-}8.473134       qo  ACTS*REG  sl4\_data1}
\CommentTok{\#\textgreater{} 3  Extraction Oceania    TOTAL {-}1.617209       qo  ACTS*REG  sl4\_data1}
\CommentTok{\#\textgreater{} 4    ProcFood Oceania    TOTAL {-}3.570224       qo  ACTS*REG  sl4\_data1}

\CommentTok{\# Return only subtotal, drop TOTAL (result is empty if there in subtotal)}
\NormalTok{data\_decomp }\OtherTok{\textless{}{-}} \FunctionTok{get\_data\_by\_var}\NormalTok{(}\StringTok{"qo"}\NormalTok{, sl4\_data1, sl4\_data2, }\AttributeTok{subtotal\_level =} \StringTok{"decomposed"}\NormalTok{)}
\FunctionTok{print}\NormalTok{(}\FunctionTok{head}\NormalTok{(data\_decomp[[}\StringTok{"sl4\_data2"}\NormalTok{]][[}\StringTok{"qo"}\NormalTok{]], }\DecValTok{4}\NormalTok{))}
\CommentTok{\#\textgreater{}            ACTS     REG   Subtotal Value Variable Dimension Experiment}
\CommentTok{\#\textgreater{} 101 GrainsCrops Oceania to changes     0       qo  ACTS*REG  sl4\_data2}
\CommentTok{\#\textgreater{} 102    MeatLstk Oceania to changes     0       qo  ACTS*REG  sl4\_data2}
\CommentTok{\#\textgreater{} 103  Extraction Oceania to changes     0       qo  ACTS*REG  sl4\_data2}
\CommentTok{\#\textgreater{} 104    ProcFood Oceania to changes     0       qo  ACTS*REG  sl4\_data2}

\CommentTok{\# Rename specific columns}
\NormalTok{data\_col\_renamed }\OtherTok{\textless{}{-}} \FunctionTok{get\_data\_by\_var}\NormalTok{(}\StringTok{"qo"}\NormalTok{, sl4\_data1, }
                             \AttributeTok{rename\_cols =} \FunctionTok{c}\NormalTok{(}\AttributeTok{REG =} \StringTok{"Region"}\NormalTok{, }\AttributeTok{COMM =} \StringTok{"Commodity"}\NormalTok{))}
\FunctionTok{str}\NormalTok{(data\_col\_renamed)}
\CommentTok{\#\textgreater{} List of 1}
\CommentTok{\#\textgreater{}  $ sl4\_data1:List of 1}
\CommentTok{\#\textgreater{}   ..$ qo:\textquotesingle{}data.frame\textquotesingle{}:   100 obs. of  7 variables:}
\CommentTok{\#\textgreater{}   .. ..$ ACTS      : chr [1:100] "GrainsCrops" "MeatLstk" "Extraction" "ProcFood" ...}
\CommentTok{\#\textgreater{}   .. ..$ Region    : chr [1:100] "Oceania" "Oceania" "Oceania" "Oceania" ...}
\CommentTok{\#\textgreater{}   .. ..$ Subtotal  : chr [1:100] "TOTAL" "TOTAL" "TOTAL" "TOTAL" ...}
\CommentTok{\#\textgreater{}   .. ..$ Value     : num [1:100] {-}5.28 {-}8.47 {-}1.62 {-}3.57 8.39 ...}
\CommentTok{\#\textgreater{}   .. ..$ Variable  : chr [1:100] "qo" "qo" "qo" "qo" ...}
\CommentTok{\#\textgreater{}   .. ..$ Dimension : chr [1:100] "ACTS*REG" "ACTS*REG" "ACTS*REG" "ACTS*REG" ...}
\CommentTok{\#\textgreater{}   .. ..$ Experiment: chr [1:100] "sl4\_data1" "sl4\_data1" "sl4\_data1" "sl4\_data1" ...}

\CommentTok{\# Rename experiment names}
\NormalTok{data\_exp\_renamed }\OtherTok{\textless{}{-}} \FunctionTok{get\_data\_by\_var}\NormalTok{(}\StringTok{"qo"}\NormalTok{, sl4\_data1, sl4\_data2, }
                             \AttributeTok{experiment\_names =} \FunctionTok{c}\NormalTok{(}\StringTok{"EXP1"}\NormalTok{, }\StringTok{"EXP2"}\NormalTok{))}
\FunctionTok{print}\NormalTok{(}\FunctionTok{names}\NormalTok{(data\_exp\_renamed))}
\CommentTok{\#\textgreater{} [1] "EXP1" "EXP2"}

\CommentTok{\# Merge variable data across multiple datasets with custom experiment names}
\NormalTok{data\_merged }\OtherTok{\textless{}{-}} \FunctionTok{get\_data\_by\_var}\NormalTok{(, sl4\_data1, sl4\_data2,}
                            \AttributeTok{experiment\_names =} \FunctionTok{c}\NormalTok{(}\StringTok{"EXP1"}\NormalTok{, }\StringTok{"EXP2"}\NormalTok{), }
                            \AttributeTok{merge\_data =} \ConstantTok{TRUE}\NormalTok{,}
                            \AttributeTok{rename\_cols =} \FunctionTok{c}\NormalTok{(}\AttributeTok{REG =} \StringTok{"Region"}\NormalTok{, }\AttributeTok{COMM =} \StringTok{"Commodity"}\NormalTok{))}
\FunctionTok{print}\NormalTok{(}\FunctionTok{head}\NormalTok{(data\_merged}\SpecialCharTok{$}\NormalTok{merged[[}\DecValTok{1}\NormalTok{]], }\DecValTok{4}\NormalTok{))}
\CommentTok{\#\textgreater{}     Commodity  Region Subtotal    Value Variable Dimension Experiment}
\CommentTok{\#\textgreater{} 1 GrainsCrops Oceania    TOTAL 10.34259      pds  COMM*REG       EXP1}
\CommentTok{\#\textgreater{} 2    MeatLstk Oceania    TOTAL 11.86026      pds  COMM*REG       EXP1}
\CommentTok{\#\textgreater{} 3  Extraction Oceania    TOTAL 11.49640      pds  COMM*REG       EXP1}
\CommentTok{\#\textgreater{} 4    ProcFood Oceania    TOTAL 13.40175      pds  COMM*REG       EXP1}
\end{Highlighting}
\end{Shaded}

Now, let's delve into
\texttt{\textless{}get\_data\_by\_dims\textgreater{}}. This command
offers an additional feature compared to
\texttt{\textless{}get\_data\_by\_var\textgreater{}}: it allows for
\texttt{patter\_mix}. For instance, REG\emph{COMM and COMM}REG can be
treated as equivalent when merging data (this only applies if
merge\_data = TRUE). You can experiment with the following commands:

\begin{Shaded}
\begin{Highlighting}[]
\CommentTok{\# Merge data by dimensions (e.g., REG*COMM != COMM*REG)}
\NormalTok{data\_no\_mix }\OtherTok{\textless{}{-}} \FunctionTok{get\_data\_by\_dims}\NormalTok{(}\ConstantTok{NULL}\NormalTok{, sl4\_data1, sl4\_data2, }
                                \AttributeTok{merge\_data =} \ConstantTok{TRUE}\NormalTok{, }
                                \AttributeTok{pattern\_mix =} \ConstantTok{FALSE}\NormalTok{)}

\CommentTok{\# Merge data while allowing interchangeable dimensions (e.g., REG*COMM = COMM*REG)}
\NormalTok{data\_pattern\_mixed }\OtherTok{\textless{}{-}} \FunctionTok{get\_data\_by\_dims}\NormalTok{(}\ConstantTok{NULL}\NormalTok{, sl4\_data1, sl4\_data2, }
                                       \AttributeTok{merge\_data =} \ConstantTok{TRUE}\NormalTok{, }
                                       \AttributeTok{pattern\_mix =} \ConstantTok{TRUE}\NormalTok{)}
\end{Highlighting}
\end{Shaded}

This flexibility is particularly useful when working with datasets where
dimension order does not affect the interpretation of the data.

\section{Grouping Data By Dimension}\label{grouping-data-by-dimension}

The \texttt{\textless{}group\_data\_by\_dims\textgreater{}} function is
a powerful tool that \textbf{categorizes extracted data into meaningful
dimension-based groups}. This is a key feature of this package, as it
allows GEMPACK users (particularly those working with GTAP model
results) to merge data into a structured and useful dataframe.

For example, if users want to retrieve all variables defined by
\texttt{REG}, they can simply set \texttt{REG} as a priority while also
assigning a new column name for \texttt{REG}, such as
\texttt{REG\ =\ "Region"}. The function will then \textbf{merge all
datasets that contain \texttt{REG} as a dimension element into a single
dataframe}, ensuring that all relevant data is consolidated.

Unlike \texttt{\textless{}get\_data\_by\_dims\textgreater{}}, which
focuses on extracting data,
\texttt{\textless{}group\_data\_by\_dims\textgreater{}}
\textbf{organizes and merges data dynamically} based on the structure
and priority defined by the user.

It is particularly useful because: - It \textbf{groups extracted data by
dimension levels} (\texttt{1D}, \texttt{2D}, \texttt{3D}, \ldots). -
Users can \textbf{define priority-based merging} (e.g., prioritizing
\texttt{REG} first, then \texttt{COMM}). The function first attempts to
merge all datasets containing \texttt{REG}, then moves on to merge
datasets containing \texttt{COMM}. If a dataset contains
\texttt{REG*COMM}, it will be included in the \texttt{REG} output. - It
\textbf{automatically renames dimensions}
(\texttt{auto\_rename\ =\ TRUE}) to improve dataset compatibility while
preserving all variable information. - It \textbf{transforms data into a
structured long format}, simplifying further analysis. - It
\textbf{supports merging grouped data across multiple datasets}. - It
\textbf{filters subtotal levels}, allowing users to retain
\texttt{"TOTAL"} values or decomposed components as needed. - If certain
datasets cannot be merged, the function generates a \textbf{detailed
report identifying the root of the issue}, enabling users to manually
manipulate the data if necessary. - This function primarily focuses on
\textbf{1D and 2D data}, ensuring that \textbf{higher-dimension data
(\textgreater2D) is merged only with datasets that share the same
pattern}.

\subsection{\texorpdfstring{\textbf{Understanding Priority List in
\texttt{group\_data\_by\_dims}}}{Understanding Priority List in group\_data\_by\_dims}}\label{understanding-priority-list-in-group_data_by_dims}

The \textbf{priority list} in
\texttt{\textless{}group\_data\_by\_dims\textgreater{}} allows users to
control \textbf{how dimensions are grouped and merged}.\\
By defining priorities, the function ensures that \textbf{datasets
containing high-priority dimensions are merged as much as possible}, but
\textbf{only if they share the same structure}.

\subsubsection{\texorpdfstring{\textbf{1. Single Priority Example
(\texttt{REG}
only)}}{1. Single Priority Example (REG only)}}\label{single-priority-example-reg-only}

If we prioritize only \texttt{REG}, the function will \textbf{attempt to
merge all datasets containing \texttt{REG} as much as possible}, while
ensuring that datasets with different structures remain separate.

\begin{Shaded}
\begin{Highlighting}[]
\CommentTok{\# Define single priority (Only Region{-}based grouping)}
\NormalTok{priority\_list }\OtherTok{\textless{}{-}} \FunctionTok{list}\NormalTok{(}\StringTok{"Region"} \OtherTok{=} \FunctionTok{c}\NormalTok{(}\StringTok{"REG"}\NormalTok{))}

\CommentTok{\# Grouping data with a single priority}
\NormalTok{grouped\_data\_single }\OtherTok{\textless{}{-}} \FunctionTok{group\_data\_by\_dims}\NormalTok{(}\StringTok{"ALL"}\NormalTok{, sl4\_data1, sl4\_data2,}
                                          \AttributeTok{priority =}\NormalTok{ priority\_list, }\AttributeTok{auto\_rename =} \ConstantTok{TRUE}\NormalTok{)}

\CommentTok{\# Print structure}
\FunctionTok{print}\NormalTok{(}\FunctionTok{names}\NormalTok{(grouped\_data\_single))}
\CommentTok{\#\textgreater{} [1] "1D" "2D" "3D" "4D"}
\FunctionTok{print}\NormalTok{(}\FunctionTok{names}\NormalTok{(grouped\_data\_single[[}\StringTok{"1D"}\NormalTok{]]))}
\CommentTok{\#\textgreater{} [1] "Region" "Other"}
\FunctionTok{print}\NormalTok{(}\FunctionTok{names}\NormalTok{(grouped\_data\_single[[}\StringTok{"2D"}\NormalTok{]]))}
\CommentTok{\#\textgreater{} [1] "Region"}
\end{Highlighting}
\end{Shaded}

What happens here? - The function tries to merge all datasets containing
REG, but only if they share the same data structure. - If some datasets
have different structures, they will remain separate. - The merging
happens separately in each data dimension (e.g., 1D, 2D, etc.).

\subsubsection{\texorpdfstring{\textbf{2. Multiple Priorities Example
(\texttt{COMM} before
\texttt{REG})}}{2. Multiple Priorities Example (COMM before REG)}}\label{multiple-priorities-example-comm-before-reg}

If we prioritize \textbf{\texttt{COMM} first and \texttt{REG} second},
datasets containing \texttt{COMM} will be \textbf{merged first},
followed by \texttt{REG}, \textbf{as much as possible} while ensuring
datasets with different structures remain separate.

\begin{Shaded}
\begin{Highlighting}[]
\CommentTok{\# Define multiple priority levels: Sector first, then Region}
\NormalTok{priority\_list }\OtherTok{\textless{}{-}} \FunctionTok{list}\NormalTok{(}
  \StringTok{"Sector"} \OtherTok{=} \FunctionTok{c}\NormalTok{(}\StringTok{"COMM"}\NormalTok{, }\StringTok{"ACTS"}\NormalTok{),}
  \StringTok{"Region"} \OtherTok{=} \FunctionTok{c}\NormalTok{(}\StringTok{"REG"}\NormalTok{)}
\NormalTok{)}

\CommentTok{\# Grouping data with multiple priorities}
\NormalTok{grouped\_data\_multiple }\OtherTok{\textless{}{-}} \FunctionTok{group\_data\_by\_dims}\NormalTok{(}\StringTok{"ALL"}\NormalTok{, sl4\_data1, }
                                            \AttributeTok{priority =}\NormalTok{ priority\_list, }
                                            \AttributeTok{auto\_rename =} \ConstantTok{TRUE}\NormalTok{)}

\CommentTok{\# Print structure}
\FunctionTok{print}\NormalTok{(}\FunctionTok{names}\NormalTok{(grouped\_data\_multiple))}
\CommentTok{\#\textgreater{} [1] "1D" "2D" "3D" "4D"}
\FunctionTok{print}\NormalTok{(}\FunctionTok{names}\NormalTok{(grouped\_data\_multiple[[}\StringTok{"1D"}\NormalTok{]]))}
\CommentTok{\#\textgreater{} [1] "Sector" "Region" "Other"}
\FunctionTok{print}\NormalTok{(}\FunctionTok{names}\NormalTok{(grouped\_data\_multiple[[}\StringTok{"2D"}\NormalTok{]]))}
\CommentTok{\#\textgreater{} [1] "Sector" "Region"}
\end{Highlighting}
\end{Shaded}

What happens here?\\
- The function attempts to merge all datasets containing \texttt{COMM}
and \texttt{ACTS} first, both of which are merged into the
\textbf{Sector} column before attempting to merge other datasets with
\texttt{REG}, as long as they share the same structure.\\
- Therefore, if datasets contain \texttt{REG*COMM} and
\texttt{REG*ACTS}, they will be merged into the \textbf{Sector}
dataframe, \textbf{not the REG dataframe}. - If datasets cannot be
merged due to structural differences, they are kept separate in their
respective dimension groups.

\subsection{\texorpdfstring{\textbf{Understanding \texttt{auto\_rename}
in
\texttt{group\_data\_by\_dims}}}{Understanding auto\_rename in group\_data\_by\_dims}}\label{understanding-auto_rename-in-group_data_by_dims}

The \texttt{auto\_rename} option in
\texttt{\textless{}group\_data\_by\_dims\textgreater{}} plays a crucial
role in ensuring \textbf{successful and smooth merging} of datasets.\\
When enabled, it \textbf{automatically renames lower-priority
dimensions} to \texttt{"Dim1"}, \texttt{"Dim2"}, etc., which allows
datasets with slightly different dimension names to be merged, while
still preserving the original dimension structure.

Without \texttt{auto\_rename}, datasets with similar but non-identical
dimension names \textbf{may fail to merge}, leading to separate outputs
instead of a unified dataset.

\subsubsection{\texorpdfstring{\textbf{Example: Grouping Without
\texttt{auto\_rename}}}{Example: Grouping Without auto\_rename}}\label{example-grouping-without-auto_rename}

By default (\texttt{auto\_rename\ =\ FALSE}), the function \textbf{keeps
all original dimension names}.\\
This means datasets that contain \textbf{similar but not identical
dimensions} (e.g., \texttt{REG*ENDW} and \texttt{REG*END})
\textbf{cannot be merged}.

\begin{Shaded}
\begin{Highlighting}[]
\CommentTok{\# Define priority: First by Sector (COMM, ACTS), then by Region (REG)}
\NormalTok{priority\_list }\OtherTok{\textless{}{-}} \FunctionTok{list}\NormalTok{(}
  \StringTok{"Sector"} \OtherTok{=} \FunctionTok{c}\NormalTok{(}\StringTok{"COMM"}\NormalTok{, }\StringTok{"ACTS"}\NormalTok{),}
  \StringTok{"Region"} \OtherTok{=} \FunctionTok{c}\NormalTok{(}\StringTok{"REG"}\NormalTok{)}
\NormalTok{)}

\CommentTok{\# Grouping data without auto\_rename}
\NormalTok{grouped\_data\_no\_rename }\OtherTok{\textless{}{-}} \FunctionTok{group\_data\_by\_dims}\NormalTok{(}\StringTok{"ALL"}\NormalTok{, sl4\_data1, }
                                             \AttributeTok{priority =}\NormalTok{ priority\_list, }
                                             \AttributeTok{auto\_rename =} \ConstantTok{FALSE}\NormalTok{)}

\CommentTok{\# Print structure}
\FunctionTok{print}\NormalTok{(}\FunctionTok{names}\NormalTok{(grouped\_data\_no\_rename))}
\CommentTok{\#\textgreater{} [1] "1D"     "2D"     "3D"     "4D"     "report"}
\FunctionTok{print}\NormalTok{(}\FunctionTok{names}\NormalTok{(grouped\_data\_no\_rename[[}\StringTok{"1D"}\NormalTok{]]))}
\CommentTok{\#\textgreater{} [1] "Sector"   "Region"   "unmerged"}
\FunctionTok{print}\NormalTok{(}\FunctionTok{names}\NormalTok{(grouped\_data\_no\_rename[[}\StringTok{"2D"}\NormalTok{]]))}
\CommentTok{\#\textgreater{} [1] "Sector"   "unmerged"}
\end{Highlighting}
\end{Shaded}

Without \texttt{auto\_rename}, the number of mergeable dataframes will
be \textbf{less} compared to when \texttt{auto\_rename\ =\ TRUE}, as it
\textbf{does not allow merging across different dimension names}, even
if their structures are otherwise compatible.

\section{Reshaping and Renaming Data}\label{reshaping-and-renaming-data}

\subsection{Pivoting Data}\label{pivoting-data}

The \texttt{\textless{}pivot\_data\textgreater{}} function transforms
long-format data from SL4 or HAR objects into a wide format, making it
more suitable for analysis and visualization. This transformation is
particularly useful when working with GEMPACK outputs that need to be
reshaped for reporting or further processing.

\begin{Shaded}
\begin{Highlighting}[]
\CommentTok{\# Extract data to pivot}
\NormalTok{data\_multiple }\OtherTok{\textless{}{-}} \FunctionTok{get\_data\_by\_var}\NormalTok{(}\FunctionTok{c}\NormalTok{(}\StringTok{"qo"}\NormalTok{, }\StringTok{"pca"}\NormalTok{), sl4\_data1)}

\CommentTok{\# Pivot a single column}
\NormalTok{pivoted\_single }\OtherTok{\textless{}{-}} \FunctionTok{pivot\_data}\NormalTok{(data\_multiple, }
                             \AttributeTok{pivot\_cols =} \StringTok{"REG"}\NormalTok{)}

\CommentTok{\# Pivot multiple columns}
\NormalTok{pivoted\_multi }\OtherTok{\textless{}{-}} \FunctionTok{pivot\_data}\NormalTok{(data\_multiple, }
                            \AttributeTok{pivot\_cols =} \FunctionTok{c}\NormalTok{(}\StringTok{"COMM"}\NormalTok{, }\StringTok{"REG"}\NormalTok{))}
\end{Highlighting}
\end{Shaded}

\subsection{Pivot Data Hierarchy}\label{pivot-data-hierarchy}

While \texttt{pivot\_data} provides basic pivoting functionality,
\texttt{pivot\_data\_hierarchy} offers enhanced capabilities for
creating hierarchical pivot tables similar to those found in spreadsheet
applications. This function is particularly useful when you need to
maintain dimensional hierarchies in your output or create Excel-ready
pivot tables.

Key differences from regular pivoting:

\begin{itemize}
\tightlist
\item
  Maintains hierarchical relationships between pivot columns
\item
  Supports direct Excel export with properly formatted headers
\item
  Preserves dimension order as specified in pivot\_cols
\item
  Creates nested column headers reflecting the hierarchy
\item
  \textbf{Must be exported to XLSX only using the command option export
  = TRUE, file\_path = your\path\out. This pivot table cannot be
  exported with the command.}
\end{itemize}

\begin{Shaded}
\begin{Highlighting}[]
\CommentTok{\# Create hierarchical pivot without export}
\NormalTok{pivot\_hier }\OtherTok{\textless{}{-}} \FunctionTok{pivot\_data\_hierarchy}\NormalTok{(data\_multiple, }
                                  \AttributeTok{pivot\_cols =} \FunctionTok{c}\NormalTok{(}\StringTok{"REG"}\NormalTok{, }\StringTok{"COMM"}\NormalTok{))}

\CommentTok{\# Create and export to Excel in one step}
\NormalTok{pivot\_export }\OtherTok{\textless{}{-}} \FunctionTok{pivot\_data\_hierarchy}\NormalTok{(data\_multiple, }
                                   \AttributeTok{pivot\_cols =} \FunctionTok{c}\NormalTok{(}\StringTok{"REG"}\NormalTok{, }\StringTok{"COMM"}\NormalTok{),}
                                   \AttributeTok{export =} \ConstantTok{TRUE}\NormalTok{,}
                                   \AttributeTok{file\_path =} \StringTok{"pivot\_output.xlsx"}\NormalTok{)}
\end{Highlighting}
\end{Shaded}

The hierarchy in the resulting pivot table follows the order specified
in \texttt{pivot\_cols}. For example, when using
\texttt{c("REG",\ "COMM")}, the output will show:

\begin{itemize}
\tightlist
\item
  First level: Region (REG)\\
\item
  Second level: Commodity (COMM) nested under each region
\item
  Values arranged according to this hierarchy
\end{itemize}

When exported to Excel, the hierarchical structure is automatically
formatted with:

\begin{itemize}
\tightlist
\item
  Properly aligned multi-level column headers
\item
  Bold formatting for header levels\\
\item
  Clean sheet names for multiple variables
\end{itemize}

\subsection{Rename Dimensions}\label{rename-dimensions}

The \texttt{rename\_dims} function provides flexible renaming
capabilities for dimensions in SL4 or HAR objects. You can rename either
dimension names, list names, or both.

\begin{Shaded}
\begin{Highlighting}[]
\CommentTok{\# Define a renaming map}
\NormalTok{mapping\_df }\OtherTok{\textless{}{-}} \FunctionTok{data.frame}\NormalTok{(}
 \AttributeTok{old =} \FunctionTok{c}\NormalTok{(}\StringTok{"REG"}\NormalTok{, }\StringTok{"COMM"}\NormalTok{),}
 \AttributeTok{new =} \FunctionTok{c}\NormalTok{(}\StringTok{"Region"}\NormalTok{, }\StringTok{"Commodity"}\NormalTok{)}
\NormalTok{)}

\CommentTok{\# Rename dimensions only}
\NormalTok{renamed\_dims }\OtherTok{\textless{}{-}} \FunctionTok{rename\_dims}\NormalTok{(sl4\_data1, mapping\_df)}

\CommentTok{\# Rename both dimensions and list names}
\NormalTok{renamed\_both }\OtherTok{\textless{}{-}} \FunctionTok{rename\_dims}\NormalTok{(sl4\_data1, mapping\_df, }\AttributeTok{rename\_list\_names =} \ConstantTok{TRUE}\NormalTok{)}
\end{Highlighting}
\end{Shaded}

The mapping dataframe must have two columns: the first for current names
and the second for new names. The function preserves data structure
while updating dimension labels according to the specified mapping.

\section{Exporting Data to
CSV/STATA/TEXT/RDS/XLSX}\label{exporting-data-to-csvstatatextrdsxlsx}

The \texttt{export\_data} function allows you to export SL4 and HAR data
into various formats. Supported formats include: \texttt{"csv"},
\texttt{"xlsx"}, \texttt{"stata"}, \texttt{"txt"}, and \texttt{"rds"}.
You can export single data frames or multi-variable results while
preserving their structure.

\begin{Shaded}
\begin{Highlighting}[]
\CommentTok{\# Export as CSV files}
\FunctionTok{export\_data}\NormalTok{(data\_multiple, }\StringTok{"output\_directory"}\NormalTok{, }
            \AttributeTok{format =} \StringTok{"csv"}\NormalTok{)}

\CommentTok{\# Export data in all available formats }
\FunctionTok{export\_data}\NormalTok{(data\_multiple, }\StringTok{"output\_directory"}\NormalTok{, }
           \AttributeTok{format =} \FunctionTok{c}\NormalTok{(}\StringTok{"csv"}\NormalTok{, }\StringTok{"xlsx"}\NormalTok{, }\StringTok{"stata"}\NormalTok{, }\StringTok{"txt"}\NormalTok{, }\StringTok{"rds"}\NormalTok{),}
           \AttributeTok{create\_subfolder =} \ConstantTok{TRUE}\NormalTok{,}
           \AttributeTok{multi\_sheet\_xlsx =} \ConstantTok{TRUE}\NormalTok{)}
\end{Highlighting}
\end{Shaded}

\section{Exploring Data Structure}\label{exploring-data-structure}

Since this package is designed to handle multiple inputs with a similar
structure, such as simulation outputs from the GTAP model with different
shocks or experiments, the first important step is to understand the
data structure. This process can also be useful even with a single
input, as it helps in analyzing the data shape and dimension size of
each variable.

There are a couple of commands available to illustrate the data
structure, all of which can be applied to both \texttt{.har} and
\texttt{.sl4} files in the same manner.

\subsection{Variable Structure}\label{variable-structure}

To get a summary of variable names, dimension counts, dimension
patterns, and optionally the column and observation counts for one or
multiple variables from a single or multiple datasets (return as
separate lists):

\begin{Shaded}
\begin{Highlighting}[]
\CommentTok{\# (1) Getting all variables from the input file}
\NormalTok{vars\_har\_sum }\OtherTok{\textless{}{-}} \FunctionTok{get\_var\_structure}\NormalTok{(}\StringTok{"ALL"}\NormalTok{, har\_data1)}
\NormalTok{vars\_sl4\_sum }\OtherTok{\textless{}{-}} \FunctionTok{get\_var\_structure}\NormalTok{(, har\_data1)}

\CommentTok{\# (2) Getting selected variables}
\NormalTok{var\_sl4\_sum }\OtherTok{\textless{}{-}} \FunctionTok{get\_var\_structure}\NormalTok{(}\FunctionTok{c}\NormalTok{(}\StringTok{"pds"}\NormalTok{,}\StringTok{"pfd"}\NormalTok{,}\StringTok{"pms"}\NormalTok{), sl4\_data1)}
\FunctionTok{print}\NormalTok{(}\FunctionTok{head}\NormalTok{(var\_sl4\_sum[[}\StringTok{"sl4\_data1"}\NormalTok{]], }\DecValTok{4}\NormalTok{))}
\CommentTok{\#\textgreater{}   Variable    Dimensions DimSize DataShape}
\CommentTok{\#\textgreater{} 1      pds      COMM*REG       2     10x10}
\CommentTok{\#\textgreater{} 2      pfd COMM*ACTS*REG       3  10x10x10}
\CommentTok{\#\textgreater{} 3      pms      COMM*REG       2     10x10}


\CommentTok{\# (3) Including column size and number of observation in the summary}
\NormalTok{var\_sl4\_sum }\OtherTok{\textless{}{-}} \FunctionTok{get\_var\_structure}\NormalTok{(}\FunctionTok{c}\NormalTok{(}\StringTok{"pds"}\NormalTok{,}\StringTok{"pfd"}\NormalTok{,}\StringTok{"pms"}\NormalTok{), sl4\_data1, sl4\_data2, }
                                 \AttributeTok{include\_col\_size =} \ConstantTok{TRUE}\NormalTok{)}
\FunctionTok{print}\NormalTok{(}\FunctionTok{head}\NormalTok{(var\_sl4\_sum[[}\StringTok{"sl4\_data1"}\NormalTok{]], }\DecValTok{4}\NormalTok{))}
\CommentTok{\#\textgreater{}   Variable    Dimensions DimSize DataShape No.Col No.Obs}
\CommentTok{\#\textgreater{} 1      pds      COMM*REG       2     10x10     10     10}
\CommentTok{\#\textgreater{} 2      pfd COMM*ACTS*REG       3  10x10x10    100     10}
\CommentTok{\#\textgreater{} 3      pms      COMM*REG       2     10x10     10     10}
\end{Highlighting}
\end{Shaded}

Understanding the data structure is crucial for aggregating data across
multiple experiments (inputs). Even when using the same variables,
different experiments may have varying column sizes or data structures
due to factors such as subtotal effects and other experimental settings.
These discrepancies can lead to errors if the merging process relies
solely on variable names without accounting for structural differences.

To compare data structures across multiple experiments, use the
following command:

\begin{Shaded}
\begin{Highlighting}[]
\CommentTok{\# (1) Comparing all variable structures across experiments }
\NormalTok{vars\_comparison }\OtherTok{\textless{}{-}} \FunctionTok{compare\_var\_structure}\NormalTok{(}
  \AttributeTok{variables =} \StringTok{"ALL"}\NormalTok{, sl4\_data1, sl4\_data2}
\NormalTok{)}
\FunctionTok{print}\NormalTok{(vars\_comparison}\SpecialCharTok{$}\NormalTok{match[}\DecValTok{1}\SpecialCharTok{:}\DecValTok{2}\NormalTok{, ])}
\CommentTok{\#\textgreater{}   Variable    Dimensions DataShape input1\_ColSize input2\_ColSize}
\CommentTok{\#\textgreater{} 1      afa COMM*ACTS*REG  10x10x10            100            100}
\CommentTok{\#\textgreater{} 2    afall COMM*ACTS*REG  10x10x10            100            100}

\CommentTok{\# (2) Comparing selected variable structures across experiments }
\NormalTok{var\_comparison }\OtherTok{\textless{}{-}} \FunctionTok{compare\_var\_structure}\NormalTok{(}
  \AttributeTok{variables =} \FunctionTok{c}\NormalTok{(}\StringTok{"pds"}\NormalTok{, }\StringTok{"pms"}\NormalTok{), sl4\_data1, sl4\_data2}
\NormalTok{)}
\FunctionTok{print}\NormalTok{(var\_comparison}\SpecialCharTok{$}\NormalTok{match[}\DecValTok{1}\SpecialCharTok{:}\DecValTok{2}\NormalTok{, ])}
\CommentTok{\#\textgreater{}   Variable Dimensions DataShape input1\_ColSize input2\_ColSize}
\CommentTok{\#\textgreater{} 1      pds   COMM*REG     10x10             10             10}
\CommentTok{\#\textgreater{} 2      pms   COMM*REG     10x10             10             10}
\end{Highlighting}
\end{Shaded}

This function returns a list containing: - \textbf{match}: A data frame
of variables with identical dimension names and structures across
inputs. - \textbf{diff} (if any): A data frame listing variables with
structural mismatches. Note: If this list appears, you may need to focus
on handling these variables. It is designed to report potential
issue-causing variables when merging datasets.

Additionally, the function includes the
\texttt{\textless{}keep\_unique\textgreater{}} option (default:
\texttt{FALSE}) which allows users to extract variables with unique
names and structures across all inputs. This is particularly useful when
inputs contain different sets of variables that need to be combined into
a final dataset.

\begin{Shaded}
\begin{Highlighting}[]
\CommentTok{\# (3) Extracting unique variable structures}
\NormalTok{unique\_vars }\OtherTok{\textless{}{-}} \FunctionTok{compare\_var\_structure}\NormalTok{(, }
\NormalTok{                                     sl4\_data1, sl4\_data2,}
  \AttributeTok{keep\_unique =} \ConstantTok{TRUE}
\NormalTok{)}
\FunctionTok{print}\NormalTok{(unique\_vars}\SpecialCharTok{$}\NormalTok{match[}\DecValTok{1}\SpecialCharTok{:}\DecValTok{10}\NormalTok{, ])}
\CommentTok{\#\textgreater{}    Variable    Dimensions DimSize DataShape}
\CommentTok{\#\textgreater{} 1       afa COMM*ACTS*REG       3  10x10x10}
\CommentTok{\#\textgreater{} 2     afall COMM*ACTS*REG       3  10x10x10}
\CommentTok{\#\textgreater{} 3     afcom          COMM       1        10}
\CommentTok{\#\textgreater{} 4       afe ENDW*ACTS*REG       3   5x10x10}
\CommentTok{\#\textgreater{} 5    afeall ENDW*ACTS*REG       3   5x10x10}
\CommentTok{\#\textgreater{} 6    afecom          ENDW       1         5}
\CommentTok{\#\textgreater{} 7    afereg           REG       1        10}
\CommentTok{\#\textgreater{} 8    afesec          ACTS       1        10}
\CommentTok{\#\textgreater{} 9     afreg           REG       1        10}
\CommentTok{\#\textgreater{} 10    afsec          ACTS       1        10}
\end{Highlighting}
\end{Shaded}

This function returns a list containing: - \textbf{match}: A data frame
displaying distinct variable structures found across inputs. -
\textbf{diff} (if any): A data frame detailing how structures differ
between inputs.

To sum up, - \texttt{keep\_unique\ =\ FALSE} → Checks whether variables
match across inputs. - \texttt{keep\_unique\ =\ TRUE} → Extracts all
unique variable structures, regardless of whether they match, while
reports any variables that do not align.

\subsection{Dimension Structure}\label{dimension-structure}

The following commands can be used to retrieve unique dimension names as
patterns and their corresponding elements:

\begin{Shaded}
\begin{Highlighting}[]
\CommentTok{\# (1) Extracting dimension patterns (e.g., REG*COMM*ACTS)}
\NormalTok{dims\_strg\_har }\OtherTok{\textless{}{-}} \FunctionTok{get\_dim\_patterns}\NormalTok{(har\_data1, har\_data2)}
\FunctionTok{print}\NormalTok{(dims\_strg\_har[}\DecValTok{1}\SpecialCharTok{:}\DecValTok{4}\NormalTok{, ])}
\CommentTok{\#\textgreater{} [1] "scalar"     "scalar"     "scalar"     "REG*COLUMN"}

\CommentTok{\# (2) Extracting unique dimension patterns (e.g., REG*COMM*ACTS)}
\NormalTok{dims\_strg\_har }\OtherTok{\textless{}{-}} \FunctionTok{get\_dim\_patterns}\NormalTok{(har\_data1, har\_data2,}
                                  \AttributeTok{keep\_unique =}\ConstantTok{TRUE}\NormalTok{)}
\FunctionTok{print}\NormalTok{(dims\_strg\_har[}\DecValTok{1}\SpecialCharTok{:}\DecValTok{4}\NormalTok{, ])}
\CommentTok{\#\textgreater{} [1] "scalar"       "REG*COLUMN"   "ALLOCEFF*REG" "REG*CTAX"}

\CommentTok{\# (2) Extracting dimension elements e.g., REG, COMM, ACTS}
\NormalTok{dims\_strg\_har\_uniq }\OtherTok{\textless{}{-}} \FunctionTok{get\_dim\_elements}\NormalTok{(har\_data1, har\_data2, }
                                       \AttributeTok{keep\_unique =}\ConstantTok{TRUE}\NormalTok{)}
\FunctionTok{print}\NormalTok{(dims\_strg\_har\_uniq[}\DecValTok{1}\SpecialCharTok{:}\DecValTok{4}\NormalTok{, ])}
\CommentTok{\#\textgreater{} [1] "REG"      "COLUMN"   "ALLOCEFF" "CTAX"}
\end{Highlighting}
\end{Shaded}

\subsection{Patterns vs.~Elements}\label{patterns-vs.-elements}

Patterns represent structured dimension names (e.g.,
\texttt{"REG*COMM*ACTS"}), while elements extract individual dimension
elements (e.g., \texttt{"REG"}, \texttt{"COMM"}, \texttt{"ACTS"}).

\end{document}
